\documentclass{article}
\usepackage{graphicx} % Required for inserting images
\usepackage{amsmath}
\usepackage{amsthm} 
\usepackage{amsfonts}
\usepackage{float}
\theoremstyle{definition}
\newtheorem*{sol}{Solution}
\usepackage{crimson}
\usepackage[a4paper,width=150mm,top=25mm,bottom=25mm]{geometry}
\usepackage{fancyhdr}

\pagestyle{fancy}
\fancyhf{}
\fancyhead[R]{Awez}
\fancyhead[L]{Exercise 2}
\fancyfoot[C]{\thepage}


\title{\textbf{Group Theory} \\ Week 3 Exercises
\\ {\large Topics : Bezout's Lemma, Fermat's and Orbit-Stabilizer Theorem, Burnside's Lemma}}
\author{Awez}
\date{}


\begin{document}

\maketitle

\section{Solutions}

\begin{sol}[\textbf{Q1.3.1}]
	Addition modulo $n$ is a binary operation on $\mathbb{Z}_n$ since it maps every element in $\mathbb{Z}_n\times \mathbb{Z}_n$ to a unique element in $\mathbb{Z}_n$. It's because by Euclid's division lemma $a+b$ can be written as $qn+r$, $r\in \mathbb{Z}_n$ and this $r$ is unique, thus $a+b\equiv r\mod{n}$ and we have a unique mapping. This operation also nakes $\mathbb{Z}_n$ into a group since
	\begin{enumerate}
		\item It's associative, \begin{align}
			      a\cdot(b\cdot c) & = a+((b+c)\mod{n})\mod{n} \\ &= (a+b+c)\mod{n}\\ &= ((a+b)\mod{n}+c)\mod{n}\\ &= (a\cdot b)\cdot c.
		      \end{align}
		\item We have an identity $e=0$ such that $a\cdot e = e\cdot a= a$ since $a+0 = 0+a \equiv a\mod{n}$.
		\item For every $a\in \mathbb{R}_n$ we have $a' = (n-a)\mod{n}$ since we have $a\cdot a' = a'\cdot a= e \equiv 0\mod{n}$.
	\end{enumerate}
\end{sol}

\begin{sol}[\textbf{Q1.3.2}]
	So let the given set of numbers be $S=\{x |x\in \mathbb{R}_n \  \gcd(x,n) = 1\}$ and the given operation be '$\cdot$'. First, the operation is a function from $S\times S\rightarrow S$ since $\forall x,y \in S$, $xy\mod{n} \in S$. This belongs to $S$ as $\gcd(x,n)=1$ and $\gcd(y,n)=1 \implies \gcd(xy,1)=1$ and it's unique because of Euclid's division lemma. Moreover
	\begin{enumerate}
		\item It's associative,
		      \begin{align}
			      (a\cdot b)\cdot c & = ab\mod{n} \cdot c     \\
			                        & = ((ab\mod{n})c)\mod{n} \\
			                        & = (abc)\mod{n}          \\
			                        & = (a(bc\mod{n}))\mod{n} \\
			                        & = a\cdot(b\cdot c)
		      \end{align}
		\item There's an identity $e=1\in S$, since $\gcd(1,n)=1$ and
		      \begin{equation}
			      a\cdot 1 = 1 \cdot a = a \mod{n} = a
		      \end{equation}
		\item For each $a\in S$,  using Bezout's lemma since $\gcd(a,n)=1$ there exists an $x$ such that $ax\equiv 1\mod{n}$. Then $x\mod{n}$ is the inverse of $a$. Since $a\cdot x = x\cdot a = 1\mod{n}$.
	\end{enumerate}
	This $S$ is known as $\mathbb{Z}^*$.
\end{sol}

\begin{sol}[\textbf{Q2.2}]
	To prove that the example $2.2$ is a valid group action, we need to verify the two group action properties:
	\begin{enumerate}
		\item For any $g,h \in G$ and $s \in S$, we must show that $(gh)\cdot s = g\cdot(h\cdot s)$.
		      \begin{align}
			      (gh)\cdot s & = (gh)(s)           \\
			                  & = g(h(s))           \\
			                  & = g\cdot(h\cdot s),
		      \end{align}
		      which holds since $g$ and $h$ are elements of the group $G$, and $\cdot$ denotes the action on $S$.
		\item For every $s \in S$, we must show that $e\cdot s = s$, where $e$ is the identity in $G$.
		      \begin{equation}
			      e\cdot s = e(s) = s,
		      \end{equation}
		      by the definition of the group action, where $e$ acts as an identity on $S$.
	\end{enumerate}
	Hence, the example $2.2$ is a valid group action.
\end{sol}

\begin{sol}[\textbf{Q2.3}]
	To prove Burnside's Lemma, let $G$ be a finite group acting on a finite set $S$. We need to count the orbits of $G$ on $S$ in two ways:
	\begin{enumerate}
		\item First, by considering the number of fixed points of each group element $g \in G$. Define $|S^g|$ as the number of elements of $S$ fixed by $g$. Then the total number of fixed points across all elements is
		      \begin{equation}
			      \sum_{g \in G} |S^g|.
		      \end{equation}
		\item Next, count the elements in each orbit. Each orbit contains exactly $|G|/|G_s|$ elements, where $G_s$ is the stabiliser of $s \in S$. Thus, the number of orbits is
		      \begin{equation}
			      \frac{1}{|G|} \sum_{g \in G} |S^g|,
		      \end{equation}
		      which proves Burnside's Lemma.
	\end{enumerate}
\end{sol}

\begin{sol}[\textbf{Q2.4}]
	We are asked to prove that the relation $\sim$ defined by $s \sim t$ if and only if there exists a $g \in G$ such that $g\cdot s = t$ is an equivalence relation.
	\begin{enumerate}
		\item \textbf{Reflexivity:} For all $s \in S$, we have $e\cdot s = s$ where $e$ is the identity element in $G$. Thus, $s \sim s$.
		\item \textbf{Symmetry:} If $s \sim t$, then there exists $g \in G$ such that $g\cdot s = t$. Since $G$ is a group, the inverse $g^{-1}$ exists, and $g^{-1}\cdot t = s$. Thus, $t \sim s$.
		\item \textbf{Transitivity:} If $s \sim t$ and $t \sim u$, then there exist $g, h \in G$ such that $g\cdot s = t$ and $h\cdot t = u$. Thus, $h(g\cdot s) = u$, which means $(hg)\cdot s = u$, so $s \sim u$.
	\end{enumerate}
	Hence, $\sim$ is an equivalence relation.
\end{sol}




\end{document}
